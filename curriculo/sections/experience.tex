% ==============================================================================
% EXPERIÊNCIA PROFISSIONAL
% ==============================================================================

\section{Experiência Profissional}

\experience{Arquiteto de Software Sênior}{Compasso UOL (Remoto)}{Abr 2022 -- Mar 2025}{
  \begin{itemize}
    \item Projetei arquitetura de microserviços Node.js/NestJS com RabbitMQ, APIs REST e padrões SOLID/Clean Architecture; governei consistência técnica entre múltiplas squads
    \item Estruturei pipelines CI/CD multi-stage no Azure DevOps (templates reutilizáveis, gates de qualidade, Docker+ACR, IaC com Bicep)
    \item Defini padrões de engenharia (versionamento, contratos REST, observabilidade) e conduzi code reviews, elevando maturidade da plataforma
  \end{itemize}
}

\experience{Programador Back-end Node.js}{Yalo (Remoto)}{Nov 2019 -- Out 2021}{
  \begin{itemize}
    \item Desenvolvi APIs REST e bots de automação para integrações com parceiros, reduzindo retrabalho operacional
    \item Modelei e otimizei esquemas PostgreSQL/MySQL, melhorando performance de consultas críticas
    \item Trabalhei diretamente com clientes para estabilizar processos e resolver gargalos técnicos
  \end{itemize}
}

\experience{Líder de Software / Visão Computacional}{Labmetro/UFSC -- Projeto VANT3D (Presencial)}{Out 2018 -- Dez 2022}{
  \begin{itemize}
    \item Liderei desenvolvimento de pipelines C++/ROS/Gazebo/OpenCV para inspeção fotogramétrica 3D de risers em plataformas offshore (Oil \& Gas)
    \item Integrei câmeras industriais, estereoscopia, gimbal tracking e drones DJI, entregando protótipos de inspeção óptica 3D
    \item Desenvolvi ambientes virtuais ROS/Gazebo para simulação de voo e validação de algoritmos de fotogrametria
  \end{itemize}
}

\vspace{10pt}

\subsection*{Experiência Anterior}
\textbf{Programador Mobile -- SCOND} (Fev 2017 -- Dez 2017): Aplicativo híbrido Ionic/AngularJS/Cordova para iOS/Android.

\textbf{Programador Web C\# -- Lugati/CIDASC} (Jan 2014 -- Out 2015): Serviços web ASP.NET/C\#, integrações entre sistemas, metodologia SCRUM.
