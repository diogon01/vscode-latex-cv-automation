% ==============================================================================
% EXPERIÊNCIA PROFISSIONAL - PYTHON/C++/IA
% ==============================================================================

\section{Experiência Profissional}

\experience{Pesquisador em Robótica e Visão Computacional}{Labmetro/UFSC -- Projeto VANT3D (Presencial)}{Out 2018 -- Dez 2022}{
  \begin{itemize}
    \item Implantei infraestrutura completa de simulação robótica com \textbf{ROS (Robot Operating System)} e \textbf{Gazebo}, permitindo testes de algoritmos de navegação autônoma antes da validação em campo
    \item \textbf{Modelei cenários 3D} para simulação de ambientes industriais, permitindo desenvolvimento e validação de algoritmos de planejamento de trajetória em ambiente seguro e controlado
    \item Desenvolvi algoritmos de \textbf{planejamento de trajetória} (Dijkstra, distâncias euclidianas) em Python para navegação de drones em ambientes industriais complexos, mitigando riscos operacionais
    \item Criei pipelines de \textbf{visão computacional} com \textbf{YOLO} e \textbf{OpenCV} para detecção de anomalias em inspeções industriais com RPAS (Remotely Piloted Aircraft Systems)
    \item Implementei projeto de inspeção industrial para \textbf{Petrobras}, validando soluções em \textbf{cenário real} e em \textbf{hardware embarcado (Jetson Nano)}, plataforma voltada para robótica
    \item Trabalhei com \textbf{deep learning} aplicado a visão computacional: treinamento de CNNs no dataset CIFAR-10, detecção de objetos e \textbf{fotogrametria 3D aplicada a Oil \& Gas}
    \item Evoluí protótipos em Python para soluções em \textbf{C++} com \textbf{CMake} e configuração via \textbf{YAML}, garantindo performance em tempo real
    \item Publiquei \textbf{4 artigos científicos} documentando resultados da pesquisa em conferências e periódicos
    \item Acumulei \textbf{+2.000h de pesquisa} em robótica, visão computacional e fotogrametria 3D
  \end{itemize}
}

\vspace{10pt}
