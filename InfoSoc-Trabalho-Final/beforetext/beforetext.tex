% ---
% Capa
% ---
% https://tex.stackexchange.com/questions/386446/how-to-fix-destination-with-the-same-identifier-namepage-a-has-been-already
% https://tex.stackexchange.com/questions/67989/pdftex-warning-has-been-referenced-but-does-not-exist-replaced-by-a-fixed-one
\hypersetup{pageanchor=false}
\PRIVATEbookmarkthis{Capa}
\imprimircapa
\hypersetup{pageanchor=true}
% ---

% ---
% Dedicatória (mantida)
% ---
\begin{dedicatoria}
    \vspace*{\fill}
    \noindent
    \begin{adjustwidth*}{}{5.5cm}
        À minha filha, \textbf{Betina Vitória}: tua luz abriu novos caminhos e me apresentou novos desafios. Foste o motivo central da escolha deste tema. Um presente divino na minha vida, que me transforma, a cada dia, em um homem melhor.
    \end{adjustwidth*}
\end{dedicatoria}
% ---

% ---
% RESUMO em português (único resumo mantido)
% ---
\setlength{\absparsep}{18pt} % ajusta o espaçamento dos parágrafos do resumo
\begin{resumo}
    \SingleSpacing
    Este trabalho examina como o \emph{machismo estrutural} conforma conflitos de papéis vividos por pais que atuam no mercado de Tecnologia da Informação (TI), posicionando-os entre as expectativas do “profissional ideal” — disponibilidade total e desempenho contínuo — e o exercício da \emph{paternidade ativa}. Adota-se uma abordagem qualitativa triangulada: (i) revisão bibliográfica sobre conflito trabalho--família, masculinidades e paternidade; (ii) diagnóstico documental de políticas públicas e normativas organizacionais relacionadas à parentalidade no trabalho; e (iii) estudo de caso exploratório de interações em rede profissional digital (LinkedIn), com análise de conteúdo dos comentários e leitura de métricas de engajamento de uma publicação autoral (aprox.\ 40 mil impressões, 380 reações e 35 comentários). Os resultados indicam a invisibilização sistemática das dificuldades enfrentadas por pais, a naturalização do papel de provedor em detrimento do cuidado, e a prevalência de ambientes laborais que penalizam simbolicamente a dedicação à família. Argumenta-se que o conflito não é meramente individual, mas efeito de arranjos de trabalho e de gênero. Como contribuição prática, propõe-se uma agenda que inclui o reforço e a ampliação de licenças parentais e políticas de cuidado voltadas a pais, a flexibilização temporal e espacial do trabalho e a criação de culturas organizacionais não punitivas à parentalidade masculina.
    
    \vspace{\onelineskip}
    \textbf{Palavras-chave}: paternidade ativa. conflito trabalho--família. machismo estrutural. tecnologia da informação. masculinidades.
\end{resumo}
% ---

{%hidelinks
    \hypersetup{hidelinks}

    % ---
    % Sumário (mantido)
    % ---
    \tableofcontents
    \cleardoublepage

}%hidelinks
% ---
