%https://github.com/SublimeText/LaTeXTools/issues/1439
%!TEX output_directory=latexcache

% ------------------------------------------------------------------------
% ------------------------------------------------------------------------
% Modelo UFSC para Trabalhos Academicos (tese de doutorado, dissertação de
% mestrado) utilizando a classe abntex2
%
% Autor: Alisson Lopes Furlani
%   Modificações:
%   - 27/08/2019: Alisson L. Furlani, add pacote 'glossaries' para listas
% ------------------------------------------------------------------------
% ------------------------------------------------------------------------

% https://tex.stackexchange.com/questions/516058/why-my-biblatex-language-is-not-changing-when-passing-the-language-on-my-documen
% https://tex.stackexchange.com/questions/385895/how-to-make-passoptionstopackage-add-the-option-as-the-last
% \PassOptionsToPackage{brazil,main=english,spanish,french}{babel}
\PassOptionsToPackage{notocbasic}{nomencl}   
\PassOptionsToPackage{main=brazil,english,spanish,french}{babel}

\documentclass[
% -- opções da classe memoir --
12pt, % tamanho da fonte
%openright, % capítulos começam em pág ímpar (insere página vazia caso preciso)
oneside, % para impressão no anverso. Oposto a twoside
a4paper, % tamanho do papel.
% -- opções da classe abntex2 --
chapter=TITLE, % títulos de capítulos convertidos em letras maiúsculas
section=TITLE, % títulos de seções convertidos em letras maiúsculas
%subsection=TITLE, % títulos de subseções convertidos em letras maiúsculas
%subsubsection=TITLE, % títulos de subsubseções convertidos em letras maiúsculas
% -- opções do pacote babel --
english, % idioma adicional para hifenização
%french, % idioma adicional para hifenização
%spanish, % idioma adicional para hifenização
brazil, % https://tex.stackexchange.com/questions/484400/changing-the-cleveref-package-language-conjunction-definition
brazilian, % https://tex.stackexchange.com/questions/516058/why-isnt-my-biblatex-language-changing-when-passing-the-language-on-my-document
]{setup/ufscthesisx}

\addbibresource{aftertext/references.bib} % Seus arquivos de referências

% https://tex.stackexchange.com/questions/516056/why-an-empty-or-not-biblatex-declaresourcemap-is-removing-my-bibliography-acces
% https://github.com/abntex/biblatex-abnt/pull/56/files
\DeclareStyleSourcemap{%% >>>2
  % This maps some fields used in abntex2cite to biblatex fields.
  \maps[datatype=bibtex]{%
    \map{%
      \step[fieldsource=conference-number,fieldtarget=number]%
      \step[fieldsource=conference-year,fieldtarget=eventdate]%
      \step[fieldsource=conference-location,fieldtarget=venue]%
      \step[fieldsource=conference-number,fieldtarget=number]%
      \step[fieldsource=org-short,fieldtarget=shortauthor]%
      \step[fieldsource=urlaccessdate,fieldtarget=urldate]%
      \step[fieldsource=year-presented,fieldtarget=eventyear]%
      \step[fieldsource=furtherresp,fieldtarget=titleaddon]%
      \step[typesource=journalpart,typetarget=supperiodical]%
    }%
    \map[overwrite=false]{%
      \step[fieldsource=reprinted-from, final]%
      \step[fieldset=related, origfieldval]%
    }%
    \map[overwrite=false]{%
      \step[fieldsource=reprinted-text, final]%
      \step[fieldset=relatedtype, fieldvalue={reprintfrom}]%
    }%
    \map{%
      \pertype{patent}% Use the organization as sourcekey for patents
      \step[fieldsource=organization, final]%
      \step[fieldset=sortkey, origfieldval]%
    }%
    \map[overwrite=false]{%
      \pertype{thesis}%
      \pertype{phdthesis}%
      \pertype{mastersthesis}%
      \pertype{monography}%
      \step[fieldset=bookpagination, fieldvalue={sheet}]%
    }%
    % remove fields that are always useless
    \map{
      \step[fieldset=abstract, null]
      \step[fieldset=pagetotal, null]
    }
    % % remove URLs for types that are primarily printed
    % \map{
    %   \pernottype{software}
    %   \pernottype{online}
    %   \pernottype{report}
    %   \pernottype{techreport}
    %   \pernottype{standard}
    %   \pernottype{manual}
    %   \pernottype{misc}
    %   \step[fieldset=url, null]
    %   \step[fieldset=urldate, null]
    % }
    \map{
      \pertype{inproceedings}
      % remove mostly redundant conference information
      \step[fieldset=venue, null]
      \step[fieldset=eventdate, null]
      \step[fieldset=eventtitle, null]
      % do not show ISBN for proceedings
      \step[fieldset=isbn, null]
      % Citavi bug
      \step[fieldset=volume, null]
    }
  }%
}% <<<2

% ---
% Informações de dados para CAPA e FOLHA DE ROSTO
% ---
\autor{Diogo Henrique Fragoso de Oliveira}

\titulo{Paternidade no mercado de TI: conflitos de papéis entre provedor, cuidador e “profissional ideal” sob o machismo estrutural}
% Caso não tenha subtítulo, mantenha comentado:
% \subtitulo{Subtítulo (se houver)}

% Professor da disciplina (usado aqui como orientador do trabalho)
\orientador{Prof. Jose Eduardo de Lucca}
% Caso tenha coorientador, descomente e preencha:
% \coorientador{Prof. XXXXXX}

% Coordenador do curso (ainda a definir, se quiser usar)
\coordenador{Prof. XXXXXX}
% Se for coordenadora mulher, comente a linha acima e use:
% \coordenador[Coordenadora]{Nome da Coordenadora}

% Ano de entrega do trabalho
\ano{2025}
% Data da entrega/entrega final (preencher depois com o dia certo)
\data{[dia] de [mês] de 2025}
% Cidade em que o trabalho é apresentado (preencher conforme teu campus)
\local{[Cidade]}

\instituicaosigla{UFSC}
\instituicao{Universidade Federal de Santa Catarina}

% Tipo de trabalho e nível (ajustado para disciplina de graduação)
\tipotrabalho{Trabalho acadêmico de graduação}
\formacao{Bacharel em Sistemas de Informação}
\nivel{graduação}
\programa{Curso de Sistemas de Informação}
% Preencher com o centro/campus correto, por ex.: "Centro Tecnológico (CTC)"
\centro{[Centro/Campus da UFSC]}

% Dados específicos da disciplina (opcionais, use se forem chamados em outros pontos do modelo)
\newcommand{\matriculaAluno}{16203891}
\newcommand{\nomeCurso}{Sistemas de Informação}
\newcommand{\disciplinaNome}{INE5621-06238 (20252) -- Informática e Sociedade}
\newcommand{\docenteResponsavel}{Prof. Jose Eduardo de Lucca}

% FIXME: Preencha com Campus XXXXXX     ou     Centro de XXXXXX
\campus{Campus Reitor João David Ferreira Lima}
\preambulo
{%
\imprimirtipotrabalho~submetida~ao~\imprimirprograma~da~\imprimirinstituicao~para~a~obtenção~do~título~de~\imprimirformacao.
}
% ---

% ---
% Configurações de aparência do PDF final
% ---
% alterando o aspecto da cor azul
\definecolor{blue}{RGB}{41,5,195}
% informações do PDF
\makeatletter
\hypersetup{
  %pagebackref=true,
  pdftitle={\@title},
  pdfauthor={\@author},
  pdfsubject={\imprimirpreambulo},
  pdfcreator={LaTeX with abnTeX2},
  pdfkeywords={ufsc, latex, abntex2},
  colorlinks=true,            % false: boxed links; true: colored links
  linkcolor=black,%blue,              % color of internal links
  citecolor=black,%blue,              % color of links to bibliography
  filecolor=black,%magenta,           % color of file links
  urlcolor=black,%blue,
  bookmarksdepth=4
}
\makeatother
% ---

% ---
% compila a lista de abreviaturas e siglas e a lista de símbolos
% ---

% Declaração das siglas
\siglalista{ABNT}{Associação Brasileira de Normas Técnicas}

% Declaração dos simbolos
\simbololista{C}{\ensuremath{C}}{Circunferência de um círculo}
\simbololista{pi}{\ensuremath{\pi}}{Número pi}
\simbololista{r}{\ensuremath{r}}{Raio de um círculo}
\simbololista{A}{\ensuremath{A}}{Área de um círculo}

% compila a lista de abreviaturas e siglas e a lista de símbolos
\makenoidxglossaries

% ---

% ---
% compila o indice
% ---
\makeindex
% ---

% ----
% Início do documento
% ----
\begin{document}

% Seleciona o idioma do documento (conforme pacotes do babel)
%\selectlanguage{english}
\selectlanguage{brazil}

% Retira espaço extra obsoleto entre as frases.
\frenchspacing

% Espaçamento 1.5 entre linhas
\OnehalfSpacing

% Corrige justificação
%\sloppy

% ----------------------------------------------------------
% ELEMENTOS PRÉ-TEXTUAIS
% ----------------------------------------------------------
% \pretextual %a macro \pretextual é acionado automaticamente no início de \begin{document}
% ---
% Capa, folha de rosto, ficha bibliografica, errata, folha de apróvação
% Dedicatória, agradecimentos, epígrafe, resumos, listas
% ---
% ---
% Capa
% ---
% https://tex.stackexchange.com/questions/386446/how-to-fix-destination-with-the-same-identifier-namepage-a-has-been-already
% https://tex.stackexchange.com/questions/67989/pdftex-warning-has-been-referenced-but-does-not-exist-replaced-by-a-fixed-one
\hypersetup{pageanchor=false}
\PRIVATEbookmarkthis{Capa}
\imprimircapa
\hypersetup{pageanchor=true}
% ---

% ---
% Dedicatória (mantida)
% ---
\begin{dedicatoria}
    \vspace*{\fill}
    \noindent
    \begin{adjustwidth*}{}{5.5cm}
        À minha filha, \textbf{Betina Vitória}: tua luz abriu novos caminhos e me apresentou novos desafios. Foste o motivo central da escolha deste tema. Um presente divino na minha vida, que me transforma, a cada dia, em um homem melhor.
    \end{adjustwidth*}
\end{dedicatoria}
% ---

% ---
% RESUMO em português (único resumo mantido)
% ---
\setlength{\absparsep}{18pt} % ajusta o espaçamento dos parágrafos do resumo
\begin{resumo}
    \SingleSpacing
    Este trabalho examina como o \emph{machismo estrutural} conforma conflitos de papéis vividos por pais que atuam no mercado de Tecnologia da Informação (TI), posicionando-os entre as expectativas do “profissional ideal” — disponibilidade total e desempenho contínuo — e o exercício da \emph{paternidade ativa}. Adota-se uma abordagem qualitativa triangulada: (i) revisão bibliográfica sobre conflito trabalho--família, masculinidades e paternidade; (ii) diagnóstico documental de políticas públicas e normativas organizacionais relacionadas à parentalidade no trabalho; e (iii) estudo de caso exploratório de interações em rede profissional digital (LinkedIn), com análise de conteúdo dos comentários e leitura de métricas de engajamento de uma publicação autoral (aprox.\ 40 mil impressões, 380 reações e 35 comentários). Os resultados indicam a invisibilização sistemática das dificuldades enfrentadas por pais, a naturalização do papel de provedor em detrimento do cuidado, e a prevalência de ambientes laborais que penalizam simbolicamente a dedicação à família. Argumenta-se que o conflito não é meramente individual, mas efeito de arranjos de trabalho e de gênero. Como contribuição prática, propõe-se uma agenda que inclui o reforço e a ampliação de licenças parentais e políticas de cuidado voltadas a pais, a flexibilização temporal e espacial do trabalho e a criação de culturas organizacionais não punitivas à parentalidade masculina.
    
    \vspace{\onelineskip}
    \textbf{Palavras-chave}: paternidade ativa. conflito trabalho--família. machismo estrutural. tecnologia da informação. masculinidades.
\end{resumo}
% ---

{%hidelinks
    \hypersetup{hidelinks}

    % ---
    % Sumário (mantido)
    % ---
    \tableofcontents
    \cleardoublepage

}%hidelinks
% ---

% ---

% ----------------------------------------------------------
% ELEMENTOS TEXTUAIS
% ----------------------------------------------------------
\textual

% ---
% 1 - Introdução
% ---
% ----------------------------------------------------------
\chapter{Introdução}
% ----------------------------------------------------------

\section{Contexto e justificativa}

O setor de Tecnologia da Informação (TI) opera sob a norma do ``profissional ideal'': disponibilidade contínua, atualização permanente e entregas constantes \cite{acker_1990}. Para homens que exercem a \emph{paternidade ativa}, esse padrão pressiona a conciliação entre o desempenho laboral e o cuidado cotidiano, produzindo tensões específicas entre os papéis de provedor, cuidador e trabalhador ``sem fricções'' \cite{greenhaus_beutell_1985,hochschild_machung_1989}. Tais tensões não são meramente individuais; são efeitos de arranjos de gênero que hierarquizam tarefas de cuidado e sustentam expectativas assimétricas sobre maternidade e paternidade \cite{connell_2005}.

No Brasil, a discussão pública e uma parcela relevante da literatura e das políticas institucionais tendem a associar a conciliação trabalho--família quase exclusivamente às mulheres, o que contribui para o \emph{apagamento} das dificuldades vividas por pais \cite{moser_pereira_2014}. Em um campo historicamente masculinizado como a TI, a combinação entre a figura do trabalhador sempre disponível e a naturalização do papel de provedor esvazia o reconhecimento da paternidade como responsabilidade de cuidado \cite{unesco_2017}. O \emph{gap} manifesta-se em dois níveis: (i) a relativa escassez de estudos centrados na experiência de pais trabalhadores --- especialmente em TI --- quando comparados ao vasto corpo sobre maternidade; e (ii) a persistência, em discursos cotidianos e organizacionais, de avaliações que deslegitimam ou minimizam a presença paterna no cuidado.

A motivação imediata deste trabalho surgiu de uma experiência pessoal, materializada em um desabafo publicado no LinkedIn sobre a dificuldade de sustentar o tripé família--estudos--trabalho na área de TI \cite{oliveira_linkedin_2025}. A repercussão da postagem, com múltiplos relatos de situações semelhantes, reforçou a percepção de que não se trata de um problema individual, mas de um fenômeno socialmente estruturado.

\section{Problema de pesquisa}

Quais efeitos do \emph{machismo estrutural} se manifestam nos conflitos de papéis vividos por pais que atuam no mercado de TI para conciliar responsabilidades profissionais e familiares, e de que modo essas dificuldades são invisibilizadas ou deslegitimadas no debate público e acadêmico --- particularmente nas interações em redes profissionais digitais?

\section{Objetivos}

\textbf{Objetivo geral}: investigar os conflitos de papéis entre provedor, cuidador e ``profissional ideal'' vividos por pais no mercado de TI, sob a lente do machismo estrutural.

\noindent\textbf{Objetivos específicos}:
\begin{itemize}
  \item revisar a literatura sobre conflito trabalho--família, masculinidades e paternidade ativa \cite{greenhaus_beutell_1985,connell_2005,lamb_2004,dermott_2008};
  \item mapear como políticas e normativas institucionais tratam (ou negligenciam) a paternidade no trabalho no contexto brasileiro \cite{ilo_2014,moser_pereira_2014};
  \item examinar discursos em rede profissional digital (LinkedIn) sobre paternidade e desempenho laboral em TI, com base em métricas e comentários de uma publicação autoral \cite{oliveira_linkedin_2025};
  \item propor agenda de práticas institucionais (licenças parentais, flexibilização temporal/espacial e cultura não punitiva à parentalidade masculina).
\end{itemize}

\section{Metodologia (resumo)}

Adota-se abordagem qualitativa triangulada: (i) revisão bibliográfica dirigida sobre conflito trabalho--família, masculinidades e paternidade; (ii) diagnóstico documental de políticas e normativas (legislação e diretrizes organizacionais) relativas à parentalidade no trabalho; e (iii) estudo de caso exploratório de interações em rede profissional digital, a partir de uma publicação autoral no LinkedIn, analisando conteúdo de comentários e métricas de engajamento (impressões, reações, compartilhamentos) \cite{oliveira_linkedin_2025}.

Para dimensionar a \emph{lacuna bibliográfica}, realizou-se um diagnóstico exploratório em cinco bases de acesso relativamente direto para estudantes de graduação: SciELO Brasil, Portal de Periódicos CAPES, DOAJ, Repositório Institucional da UFSC e ScienceDirect. Em todas elas aplicaram-se, de forma simétrica, combinações de descritores em português e em inglês (``maternidade e trabalho'' / ``motherhood and work'' e ``paternidade e trabalho'' / ``fatherhood and work''), no período de 2000 a 2025, contabilizando o número de registros e as proporções entre textos centrados na maternidade e na paternidade. Não se trata de uma revisão sistemática, mas de uma sondagem curta voltada a evidenciar a direção do desequilíbrio entre os dois conjuntos.

O recorte empírico foca o setor de TI. Os depoimentos oriundos do LinkedIn são apresentados de forma sintética, com preservação de anonimato quando necessário e atenção aos limites de representatividade de uma única rede social profissional.

\section{Estrutura do trabalho}

O Capítulo~2 apresenta o referencial teórico, discutindo gênero e divisão sexual do trabalho, paternidade ativa, machismo estrutural, conflito trabalho--família e características específicas do campo de TI. O Capítulo~3 realiza o diagnóstico do apagamento da paternidade ativa em políticas e literatura, articulando análise de marcos legais, panorama bibliométrico exploratório e o estudo das interações em rede profissional digital (LinkedIn). O Capítulo~4 traz as conclusões e considerações finais, retomando o problema de pesquisa, sintetizando os achados, discutindo implicações para o campo de TI e para Informática e Sociedade e indicando limitações e agenda futura.

% ---

% ---
% 2 - Capítulo 2
% ---
% ----------------------------------------------------------
\chapter{Referencial teórico}
% ----------------------------------------------------------

\section{Gênero e divisão sexual do trabalho}
A literatura sociológica tem demonstrado que o trabalho é atravessado por relações de gênero que distribuem e hierarquizam tarefas, competências e expectativas \cite{hirata_2002, saffioti_2004}. Tal distribuição sustenta o imaginário do provedor masculino e da cuidadora feminina, naturalizando a presença das mulheres no espaço doméstico e a dos homens no provimento financeiro. No plano simbólico, o \emph{habitus} de gênero reforça disposições que tornam ``evidentes'' (portanto pouco problematizadas) tais assimetrias \cite{bourdieu_2001}. Em consequência, o cuidado aparece como atributo quase exclusivo da maternidade, enquanto a paternidade tende a ser representada como apoio eventual ou complementar.

\section{Machismo estrutural e o \textit{trabalhador ideal}}
O conceito de \emph{machismo estrutural} aponta que assimetrias entre homens e mulheres não se limitam a atitudes individuais, mas se organizam em práticas institucionais, regras e rotinas que produzem resultados desiguais \cite{connell_2005}. No interior das organizações, a figura do \emph{trabalhador ideal} --- sempre disponível, sem ``interferências'' familiares --- opera como norma de desempenho e presença \cite{acker_1990}. Essa norma pressiona todas as pessoas, mas recai de modo particular sobre homens que desejam exercer a paternidade de forma ativa, pois os coloca diante de um dilema: para corresponder ao ideal, é preciso neutralizar o cuidado; para cuidar, é preciso ``falhar'' no ideal \cite{hochschild_machung_1989}.

\section{Paternidade ativa e masculinidades}
Os estudos contemporâneos sobre masculinidades e paternidade deslocam o pai da figura exclusivamente provedora para a de sujeito de cuidado, com efeitos positivos sobre desenvolvimento infantil e vínculos afetivos \cite{lamb_2004, dermott_2008}. Em paralelo, pesquisas mostram que a mudança desejada na prática paterna depende de arranjos institucionais (licenças, horários, cultura) que legitimem o engajamento dos homens no cuidado \cite{gatrell_2013}. Na ausência de tais suportes, prevalecem barreiras simbólicas (\emph{estigmas}) e materiais (sanções de carreira), que desincentivam a paternidade ativa.

\section{Conflito trabalho--família}
A teoria dos papéis fornece base para compreender tensões entre demandas do trabalho e da família \cite{greenhaus_beutell_1985}. Em contextos que exigem alta disponibilidade temporal e emocional, o conflito tende a aumentar e a produzir \emph{trade-offs} entre desempenho esperado e presença no cuidado. Embora este campo de estudos seja amplo, a maior parte da produção concentra-se na maternidade, deixando menos visíveis as experiências dos pais \cite{moser_pereira_2014}. Essa lacuna também pode ser compreendida como efeito do machismo estrutural: o cuidado é lido como ``tema de mulheres'', o que reduz a legitimidade social do pai cuidador.

\section{TI como campo masculinizado}
A área de Tecnologia da Informação foi historicamente construída como um campo masculinizado, tanto em termos de composição demográfica quanto de cultura organizacional \cite{unesco_2017}. A ênfase na disponibilidade total (\emph{on-call}, plantões, prazos apertados), na atualização constante e na dedicação extensiva cria fricções adicionais para quem exerce cuidados cotidianos. Assim, a norma do \emph{trabalhador ideal} adquire intensidade específica em TI, amplificando barreiras simbólicas e materiais para a paternidade ativa.

\section{Síntese}
O referencial sustenta três proposições: (i) as tensões vividas por pais trabalhadores em TI derivam da articulação entre a norma do \emph{trabalhador ideal} e o machismo estrutural que deslegitima o cuidado masculino; (ii) a paternidade ativa requer suporte institucional (licenças, horários, cultura) e reconhecimento simbólico para tornar-se viável sem sanções; e (iii) a relativa escassez de estudos centrados em pais, sobretudo no setor de TI, é parte do problema e deve ser evidenciada empiricamente.

\section{Nota metodológica sobre o diagnóstico bibliográfico}
Para tornar essa lacuna mais visível, o Capítulo~\ref{chap:diagnostico} realiza um diagnóstico bibliográfico exploratório em bases de acesso direto. Comparam-se, de forma simétrica, as ocorrências, em português, das combinações ``maternidade e trabalho'' e ``paternidade e trabalho'' em três bases com perfis complementares: SciELO Brasil, Portal de Periódicos da CAPES e DOAJ (\emph{Directory of Open Access Journals}), no período de 2000 a 2025.

O procedimento resume-se a contar o número de registros para cada combinação de descritores em cada base e calcular a razão entre os conjuntos centrados em maternidade e em paternidade. A estratégia não pretende esgotar a produção existente, mas oferecer um indicador simples da sub-representação da paternidade na literatura sobre trabalho--família, articulando esses resultados exploratórios às discussões teóricas apresentadas neste capítulo.

% ---

% ---
% 3 - Capítulo 3
% ---
% ----------------------------------------------------------
\chapter{Diagnóstico: apagamento da paternidade ativa em políticas e literatura}
\label{chap:diagnostico}
% ----------------------------------------------------------

\section{Políticas públicas e marcos legais no Brasil}

O ordenamento jurídico brasileiro incorpora dispositivos de proteção à maternidade
e à família, porém a paternidade permanece, em geral, tratada de modo mais restrito.
A Constituição Federal assegura a licença-maternidade como direito social
\cite{cf_1988_art7}, enquanto a licença-paternidade foi normatizada de forma mais
sucinta no Ato das Disposições Constitucionais Transitórias \cite{adct_art10_par1}. O
Programa Empresa Cidadã ampliou a licença-paternidade para empresas aderentes
\cite{lei_11770_2008_empresa_cidada,decreto_8737_2016}, e o Marco Legal da
Primeira Infância introduziu princípios e diretrizes para atenção ao cuidado nos
primeiros anos de vida \cite{lei_13257_2016_marco_primeira_infancia}. Ainda assim,
nas políticas e regulações trabalhistas, a ênfase normativa e a linguagem permanecem
majoritariamente centradas na maternidade, o que contribui para a invisibilização da
paternidade como responsabilidade de cuidado contínuo \cite{moser_pereira_2014}.

\section{Normas organizacionais e práticas em TI}

No nível organizacional, observa-se heterogeneidade: algumas empresas de TI
oferecem licenças parentais estendidas, políticas de flexibilidade temporal/espacial e
programas de apoio ao cuidado; outras mantêm arranjos que reproduzem a figura do
\emph{trabalhador ideal} sempre disponível \cite{acker_1990}. Iniciativas voluntárias e
guias corporativos avançam em linguagem inclusiva de parentalidade, mas ainda há
lacunas na implementação, monitoramento e não punição implícita de trajetórias
masculinas que priorizam o cuidado \cite{mm360_paternidade_ativa,
roberthalf_2023_paternidade_ativa}. Em ambientes de alta demanda, plantões e
atualizações constantes, tais lacunas pressionam particularmente pais que buscam
exercer a paternidade ativa, sobretudo em equipes enxutas e projetos com prazos
rígidos.

\subsection{Panorama bibliométrico da literatura (2000--2025)}

Para tornar mais visível a sub-representação da paternidade no campo trabalho--família, realizou-se um diagnóstico bibliométrico exploratório em três bases de acesso relativamente simples para estudantes de graduação: SciELO Brasil, Portal de Periódicos da CAPES e DOAJ. Em todas elas aplicaram-se, de forma simétrica, as buscas em português ``maternidade e trabalho'' e ``paternidade e trabalho'', no período de 2000 a 2025, contabilizando apenas o número bruto de registros retornados. Trata-se, portanto, de uma sondagem curta, mas suficiente para sugerir a direção do desequilíbrio.

Os resultados apontam um descompasso expressivo entre os dois conjuntos. No SciELO Brasil, foram localizados 313 registros que articulam maternidade e trabalho, contra 53 sobre paternidade e trabalho; no Portal de Periódicos da CAPES, 1\,615 contra 203; e, no DOAJ, 671 contra 101. Considerando o conjunto das três bases, isso significa que algo entre 85\% e 90\% dos textos identificados tomam a maternidade como foco, enquanto a paternidade aparece em um intervalo residual de cerca de 10\% a 15\%. Em termos intuitivos, para cada artigo sobre pais e trabalho, há entre seis e oito artigos sobre mães e trabalho.

Esse panorama não pretende descrever de forma exaustiva a produção acadêmica sobre o tema, mas funciona como um indicador simples da direção do viés: mesmo em bases amplas e de acesso aberto, a experiência de pais trabalhadores ocupa um espaço pequeno, quase marginal, quando comparada à centralidade da maternidade. Essa assimetria reforça o argumento de que o apagamento da paternidade ativa não é apenas uma impressão individual, mas um efeito concreto da forma como o campo de estudos trabalho--família foi historicamente organizado, especialmente no recorte brasileiro e latino-americano.


\begin{table}[ht]
\centering
\caption{Buscas exploratórias com descritores simétricos em bases gerais, institucionais e internacional (2000--2025)}
\label{tab:exploratorio_m_p_trabalho}

\setlength{\tabcolsep}{8pt}
\renewcommand{\arraystretch}{1.2}

\rowcolors{2}{gray!10}{white}

\begin{tabular}{p{0.40\textwidth}rrr}
\hline
\rowcolor{white}
Base &
\multicolumn{1}{c}{Maternidade (M)} &
\multicolumn{1}{c}{Paternidade (P)} &
\multicolumn{1}{c}{M:P (vezes)} \\
\hline
ScienceDirect (internacional)  & 16\,022 & 2\,829 & 5{,}7:1 \\
Periódicos CAPES               &  1\,615 &   203 & 8{,}0:1 \\
DOAJ                           &    671 &   101 & 6{,}6:1 \\
SciELO Brasil                  &    313 &    53 & 5{,}9:1 \\
Repositório Institucional UFSC &    144 &    47 & 3{,}1:1 \\
\hline
\end{tabular}
\end{table}








Em todas as bases, o número de publicações que articulam maternidade e trabalho
é muito superior ao de publicações que articulam paternidade e trabalho. No SciELO
Brasil, as mães concentram cerca de 85,5\% dos registros combinados; no Portal de
Periódicos da CAPES, aproximadamente 88,8\%; e, no DOAJ, cerca de 86,9\%.
Assim, a produção sobre maternidade e trabalho representa, de forma consistente,
algo entre 85\% e 90\% do conjunto, ao passo que a paternidade ocupa um intervalo
residual entre 10\% e 15\%.

\section{Relatos de pais cuidadores em TI (LinkedIn)}

O diagnóstico bibliométrico dialoga com evidências qualitativas observadas em
uma interação em rede social profissional. Em agosto de 2025, o autor publicou um
relato no LinkedIn sobre a dificuldade de conciliar paternidade ativa e procura de
trabalho em TI. A postagem alcançou cerca de 40 mil impressões e gerou dezenas de
comentários de profissionais que se reconheceram na situação descrita.

Um dos comentários, de um \emph{Agile Master} brasileiro, relata a ``cobrança
silenciosa de estar sempre disponível'' desde o nascimento do filho. Ao precisar
levá-lo ao médico para um cuidado de rotina, reorganizou a agenda e avisou a equipe
com antecedência; ainda assim, recebeu como feedback que deveria se organizar de
modo que a mãe pudesse levar sozinha, para não ``deixar o time na mão''. Ele
destaca a sensação de que, quando o pai cuida, isso é tratado como exceção que
precisa ser justificada, como se dividir o cuidado fosse sinal de falta de compromisso.

Outro depoimento, de um gerente de projetos e analista de negócios com
experiência internacional, indica que a situação não se restringe ao Brasil. O
profissional descreve situações vividas no Canadá em que se sentiu julgado por
priorizar o cuidado com as filhas em relação a entregas que não eram urgentes,
mencionando que ainda se espera que o homem mantenha um certo
``distanciamento'' em relação ao cuidado cotidiano com os filhos.

Um terceiro comentário, de um desenvolvedor de software pai de uma criança com
doença renal crônica, relata a perda do emprego em meio à rotina intensa de
cuidados: adaptação do trabalho remoto, acompanhamento do filho em hemodiálise,
consultas médicas e deslocamentos diários à escola, além do preparo de alimentação
em casa. Com a esposa em regime híbrido, ele assume a maior parte do cuidado
direto e, após o desligamento, passa a combinar a busca por recolocação com a
atualização profissional, afirmando-se como ``pai presente de sempre, independente
do que aconteça''.

Esses relatos não constituem amostra representativa do setor, mas funcionam como
vinhetas empíricas que ilustram a persistência do ideal de \emph{trabalhador sempre
disponível} em ambientes de TI, bem como a tendência a tratar a paternidade ativa
como exceção ou problema de agenda. A interpretação mais ampla desses achados é
desenvolvida no Capítulo~\ref{chap:conclusoes}.

% ---

% ---
% 4 - Conclusão
% ---
%\phantompart
% ----------------------------------------------------------
\chapter{Conclusões e considerações finais}
\label{chap:conclusoes}
% ----------------------------------------------------------

\section{Retomada do problema e do percurso}

Este trabalho nasceu de um desabafo que publiquei no LinkedIn\cite{oliveira_linkedin_2025},
ao relatar a dificuldade concreta de sustentar o tripé família--estudos--trabalho
na área de TI. A rotina de cuidado com minha filha pequena passou a conviver com
a cobrança de disponibilidade quase ilimitada no emprego e nas entrevistas. A
resposta à postagem – com muitos comentários de pais e mães em situações
parecidas – mostrou que o problema ia além da minha experiência individual.

A partir daí, formulei a pergunta central: de que modo o machismo estrutural e o
ideal de trabalhador sempre disponível contribuem para apagar a paternidade
ativa, em especial na área de tecnologia da informação? Para discutir essa
questão, combinei análise de marcos legais e políticas de parentalidade, revisão
de literatura sobre normas organizacionais em TI, um diagnóstico bibliométrico
exploratório em bases de dados e o exame qualitativo das interações geradas por
essa postagem no LinkedIn\cite{oliveira_linkedin_2025}.

\section{Síntese dos achados e implicações}

Do ponto de vista jurídico, o ordenamento brasileiro protege de forma muito mais
robusta a maternidade do que a paternidade: licença-maternidade, estabilidade
gestante e políticas específicas estão consolidadas, enquanto a licença-paternidade
permanece curta e fragmentada \cite{cf_1988_art7,adct_art10_par1,
lei_11770_2008_empresa_cidada,decreto_8737_2016,lei_13257_2016_marco_primeira_infancia}.
No nível organizacional, a literatura sobre TI aponta a permanência do ideal de
\emph{trabalhador ideal} sempre disponível \cite{acker_1990}, e estudos recentes
indicam que homens que explicitam o cuidado ainda enfrentam desconfiança e
punições sutis \cite{mm360_paternidade_ativa,roberthalf_2023_paternidade_ativa}.

O diagnóstico bibliométrico em cinco bases (ScienceDirect, Portal de Periódicos
CAPES, DOAJ, SciELO Brasil e Repositório Institucional da UFSC) reforçou essa
assimetria: em todas elas há de três a oito textos sobre maternidade e trabalho
para cada texto sobre paternidade e trabalho. Em outras palavras, a experiência
de pais trabalhadores aparece como tema residual, mesmo em bases amplas e
internacionais. Os relatos da postagem no LinkedIn dão rosto a esses números:
pais que levam filhos ao médico, acompanham terapias ou reorganizam horários
relatam perda de oportunidades, cobranças de disponibilidade total e a sensação
de que precisam justificar o fato de cuidar.

Para o campo de TI e para Informática e Sociedade, esses achados indicam que
tecnologias, políticas de trabalho remoto e critérios de desempenho não são
neutros: articulam-se com normas de gênero que naturalizam o homem como provedor
sempre disponível e a mulher como cuidadora principal. Nesse contexto, exercer
paternidade ativa em TI ainda tende a ser visto como exceção, e não como forma
legítima de organizar a vida profissional.

\section{Limitações e agenda futura}

Este estudo é exploratório. As buscas bibliográficas concentraram-se em cinco
bases de acesso relativamente direto, com descritores simples em português e
inglês, o que pode ter deixado de fora produções que utilizam outras
terminologias ou recortes. Do ponto de vista qualitativo, a análise se baseia em
uma única postagem no LinkedIn, condicionada à minha rede de contatos e ao
algoritmo da plataforma, e não pretende representar a diversidade do setor de TI.

Ainda assim, os resultados sugerem caminhos para pesquisas futuras e para a
prática profissional: investigar políticas internas de empresas de tecnologia sob
a perspectiva dos pais cuidadores; comparar contextos nacionais com diferentes
regimes de licença parental; e discutir como arranjos de trabalho em TI podem
incorporar o cuidado – de mães e de pais – como uma dimensão legítima da
trajetória profissional, e não como exceção a ser tolerada.

% ---

% ----------------------------------------------------------
% ELEMENTOS PÓS-TEXTUAIS
% ----------------------------------------------------------
%\postextual
% ----------------------------------------------------------

% ----------------------------------------------------------
% Referências bibliográficas
% ----------------------------------------------------------
\begingroup
  \SingleSpacing
  \setlength\bibitemsep{\baselineskip}
  \printbibliography[title=REFERÊNCIAS]
\endgroup

% ----------------------------------------------------------
% Glossário
% ----------------------------------------------------------
%
% Consulte o manual da classe abntex2 para orientações sobre o glossário.
%
%\glossary

% ----------------------------------------------------------
% Apêndices
% ----------------------------------------------------------

% ---
% Inicia os apêndices
% ---
\begin{apendicesenv}
  % \partapendices*
  \input{aftertext/apendice_a}
\end{apendicesenv}
% ---


% ----------------------------------------------------------
% Anexos
% ----------------------------------------------------------

% ---
% Inicia os anexos
% ---
\begin{anexosenv}
  % \partanexos*
  \input{aftertext/anexo_a}
\end{anexosenv}

%---------------------------------------------------------------------
% INDICE REMISSIVO
%---------------------------------------------------------------------
%\phantompart
%\printindex
%---------------------------------------------------------------------

\end{document}
