% ----------------------------------------------------------
\chapter{Diagnóstico: apagamento da paternidade ativa em políticas e literatura}
\label{chap:diagnostico}
% ----------------------------------------------------------

\section{Políticas públicas e marcos legais no Brasil}

O ordenamento jurídico brasileiro incorpora dispositivos de proteção à maternidade
e à família, porém a paternidade permanece, em geral, tratada de modo mais restrito.
A Constituição Federal assegura a licença-maternidade como direito social
\cite{cf_1988_art7}, enquanto a licença-paternidade foi normatizada de forma mais
sucinta no Ato das Disposições Constitucionais Transitórias \cite{adct_art10_par1}. O
Programa Empresa Cidadã ampliou a licença-paternidade para empresas aderentes
\cite{lei_11770_2008_empresa_cidada,decreto_8737_2016}, e o Marco Legal da
Primeira Infância introduziu princípios e diretrizes para atenção ao cuidado nos
primeiros anos de vida \cite{lei_13257_2016_marco_primeira_infancia}. Ainda assim,
nas políticas e regulações trabalhistas, a ênfase normativa e a linguagem permanecem
majoritariamente centradas na maternidade, o que contribui para a invisibilização da
paternidade como responsabilidade de cuidado contínuo \cite{moser_pereira_2014}.

\section{Normas organizacionais e práticas em TI}

No nível organizacional, observa-se heterogeneidade: algumas empresas de TI
oferecem licenças parentais estendidas, políticas de flexibilidade temporal/espacial e
programas de apoio ao cuidado; outras mantêm arranjos que reproduzem a figura do
\emph{trabalhador ideal} sempre disponível \cite{acker_1990}. Iniciativas voluntárias e
guias corporativos avançam em linguagem inclusiva de parentalidade, mas ainda há
lacunas na implementação, monitoramento e não punição implícita de trajetórias
masculinas que priorizam o cuidado \cite{mm360_paternidade_ativa,
roberthalf_2023_paternidade_ativa}. Em ambientes de alta demanda, plantões e
atualizações constantes, tais lacunas pressionam particularmente pais que buscam
exercer a paternidade ativa, sobretudo em equipes enxutas e projetos com prazos
rígidos.

\subsection{Panorama bibliométrico da literatura (2000--2025)}

Para tornar mais visível a sub-representação da paternidade no campo trabalho--família, realizou-se um diagnóstico bibliométrico exploratório em três bases de acesso relativamente simples para estudantes de graduação: SciELO Brasil, Portal de Periódicos da CAPES e DOAJ. Em todas elas aplicaram-se, de forma simétrica, as buscas em português ``maternidade e trabalho'' e ``paternidade e trabalho'', no período de 2000 a 2025, contabilizando apenas o número bruto de registros retornados. Trata-se, portanto, de uma sondagem curta, mas suficiente para sugerir a direção do desequilíbrio.

Os resultados apontam um descompasso expressivo entre os dois conjuntos. No SciELO Brasil, foram localizados 313 registros que articulam maternidade e trabalho, contra 53 sobre paternidade e trabalho; no Portal de Periódicos da CAPES, 1\,615 contra 203; e, no DOAJ, 671 contra 101. Considerando o conjunto das três bases, isso significa que algo entre 85\% e 90\% dos textos identificados tomam a maternidade como foco, enquanto a paternidade aparece em um intervalo residual de cerca de 10\% a 15\%. Em termos intuitivos, para cada artigo sobre pais e trabalho, há entre seis e oito artigos sobre mães e trabalho.

Esse panorama não pretende descrever de forma exaustiva a produção acadêmica sobre o tema, mas funciona como um indicador simples da direção do viés: mesmo em bases amplas e de acesso aberto, a experiência de pais trabalhadores ocupa um espaço pequeno, quase marginal, quando comparada à centralidade da maternidade. Essa assimetria reforça o argumento de que o apagamento da paternidade ativa não é apenas uma impressão individual, mas um efeito concreto da forma como o campo de estudos trabalho--família foi historicamente organizado, especialmente no recorte brasileiro e latino-americano.


\begin{table}[ht]
\centering
\caption{Buscas exploratórias com descritores simétricos em bases gerais, institucionais e internacional (2000--2025)}
\label{tab:exploratorio_m_p_trabalho}

\setlength{\tabcolsep}{8pt}
\renewcommand{\arraystretch}{1.2}

\rowcolors{2}{gray!10}{white}

\begin{tabular}{p{0.40\textwidth}rrr}
\hline
\rowcolor{white}
Base &
\multicolumn{1}{c}{Maternidade (M)} &
\multicolumn{1}{c}{Paternidade (P)} &
\multicolumn{1}{c}{M:P (vezes)} \\
\hline
ScienceDirect (internacional)  & 16\,022 & 2\,829 & 5{,}7:1 \\
Periódicos CAPES               &  1\,615 &   203 & 8{,}0:1 \\
DOAJ                           &    671 &   101 & 6{,}6:1 \\
SciELO Brasil                  &    313 &    53 & 5{,}9:1 \\
Repositório Institucional UFSC &    144 &    47 & 3{,}1:1 \\
\hline
\end{tabular}
\end{table}








Em todas as bases, o número de publicações que articulam maternidade e trabalho
é muito superior ao de publicações que articulam paternidade e trabalho. No SciELO
Brasil, as mães concentram cerca de 85,5\% dos registros combinados; no Portal de
Periódicos da CAPES, aproximadamente 88,8\%; e, no DOAJ, cerca de 86,9\%.
Assim, a produção sobre maternidade e trabalho representa, de forma consistente,
algo entre 85\% e 90\% do conjunto, ao passo que a paternidade ocupa um intervalo
residual entre 10\% e 15\%.

\section{Relatos de pais cuidadores em TI (LinkedIn)}

O diagnóstico bibliométrico dialoga com evidências qualitativas observadas em
uma interação em rede social profissional. Em agosto de 2025, o autor publicou um
relato no LinkedIn sobre a dificuldade de conciliar paternidade ativa e procura de
trabalho em TI. A postagem alcançou cerca de 40 mil impressões e gerou dezenas de
comentários de profissionais que se reconheceram na situação descrita.

Um dos comentários, de um \emph{Agile Master} brasileiro, relata a ``cobrança
silenciosa de estar sempre disponível'' desde o nascimento do filho. Ao precisar
levá-lo ao médico para um cuidado de rotina, reorganizou a agenda e avisou a equipe
com antecedência; ainda assim, recebeu como feedback que deveria se organizar de
modo que a mãe pudesse levar sozinha, para não ``deixar o time na mão''. Ele
destaca a sensação de que, quando o pai cuida, isso é tratado como exceção que
precisa ser justificada, como se dividir o cuidado fosse sinal de falta de compromisso.

Outro depoimento, de um gerente de projetos e analista de negócios com
experiência internacional, indica que a situação não se restringe ao Brasil. O
profissional descreve situações vividas no Canadá em que se sentiu julgado por
priorizar o cuidado com as filhas em relação a entregas que não eram urgentes,
mencionando que ainda se espera que o homem mantenha um certo
``distanciamento'' em relação ao cuidado cotidiano com os filhos.

Um terceiro comentário, de um desenvolvedor de software pai de uma criança com
doença renal crônica, relata a perda do emprego em meio à rotina intensa de
cuidados: adaptação do trabalho remoto, acompanhamento do filho em hemodiálise,
consultas médicas e deslocamentos diários à escola, além do preparo de alimentação
em casa. Com a esposa em regime híbrido, ele assume a maior parte do cuidado
direto e, após o desligamento, passa a combinar a busca por recolocação com a
atualização profissional, afirmando-se como ``pai presente de sempre, independente
do que aconteça''.

Esses relatos não constituem amostra representativa do setor, mas funcionam como
vinhetas empíricas que ilustram a persistência do ideal de \emph{trabalhador sempre
disponível} em ambientes de TI, bem como a tendência a tratar a paternidade ativa
como exceção ou problema de agenda. A interpretação mais ampla desses achados é
desenvolvida no Capítulo~\ref{chap:conclusoes}.
