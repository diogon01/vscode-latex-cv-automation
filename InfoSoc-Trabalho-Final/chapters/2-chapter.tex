% ----------------------------------------------------------
\chapter{Referencial teórico}
% ----------------------------------------------------------

\section{Gênero e divisão sexual do trabalho}
A literatura sociológica tem demonstrado que o trabalho é atravessado por relações de gênero que distribuem e hierarquizam tarefas, competências e expectativas \cite{hirata_2002, saffioti_2004}. Tal distribuição sustenta o imaginário do provedor masculino e da cuidadora feminina, naturalizando a presença das mulheres no espaço doméstico e a dos homens no provimento financeiro. No plano simbólico, o \emph{habitus} de gênero reforça disposições que tornam ``evidentes'' (portanto pouco problematizadas) tais assimetrias \cite{bourdieu_2001}. Em consequência, o cuidado aparece como atributo quase exclusivo da maternidade, enquanto a paternidade tende a ser representada como apoio eventual ou complementar.

\section{Machismo estrutural e o \textit{trabalhador ideal}}
O conceito de \emph{machismo estrutural} aponta que assimetrias entre homens e mulheres não se limitam a atitudes individuais, mas se organizam em práticas institucionais, regras e rotinas que produzem resultados desiguais \cite{connell_2005}. No interior das organizações, a figura do \emph{trabalhador ideal} --- sempre disponível, sem ``interferências'' familiares --- opera como norma de desempenho e presença \cite{acker_1990}. Essa norma pressiona todas as pessoas, mas recai de modo particular sobre homens que desejam exercer a paternidade de forma ativa, pois os coloca diante de um dilema: para corresponder ao ideal, é preciso neutralizar o cuidado; para cuidar, é preciso ``falhar'' no ideal \cite{hochschild_machung_1989}.

\section{Paternidade ativa e masculinidades}
Os estudos contemporâneos sobre masculinidades e paternidade deslocam o pai da figura exclusivamente provedora para a de sujeito de cuidado, com efeitos positivos sobre desenvolvimento infantil e vínculos afetivos \cite{lamb_2004, dermott_2008}. Em paralelo, pesquisas mostram que a mudança desejada na prática paterna depende de arranjos institucionais (licenças, horários, cultura) que legitimem o engajamento dos homens no cuidado \cite{gatrell_2013}. Na ausência de tais suportes, prevalecem barreiras simbólicas (\emph{estigmas}) e materiais (sanções de carreira), que desincentivam a paternidade ativa.

\section{Conflito trabalho--família}
A teoria dos papéis fornece base para compreender tensões entre demandas do trabalho e da família \cite{greenhaus_beutell_1985}. Em contextos que exigem alta disponibilidade temporal e emocional, o conflito tende a aumentar e a produzir \emph{trade-offs} entre desempenho esperado e presença no cuidado. Embora este campo de estudos seja amplo, a maior parte da produção concentra-se na maternidade, deixando menos visíveis as experiências dos pais \cite{moser_pereira_2014}. Essa lacuna também pode ser compreendida como efeito do machismo estrutural: o cuidado é lido como ``tema de mulheres'', o que reduz a legitimidade social do pai cuidador.

\section{TI como campo masculinizado}
A área de Tecnologia da Informação foi historicamente construída como um campo masculinizado, tanto em termos de composição demográfica quanto de cultura organizacional \cite{unesco_2017}. A ênfase na disponibilidade total (\emph{on-call}, plantões, prazos apertados), na atualização constante e na dedicação extensiva cria fricções adicionais para quem exerce cuidados cotidianos. Assim, a norma do \emph{trabalhador ideal} adquire intensidade específica em TI, amplificando barreiras simbólicas e materiais para a paternidade ativa.

\section{Síntese}
O referencial sustenta três proposições: (i) as tensões vividas por pais trabalhadores em TI derivam da articulação entre a norma do \emph{trabalhador ideal} e o machismo estrutural que deslegitima o cuidado masculino; (ii) a paternidade ativa requer suporte institucional (licenças, horários, cultura) e reconhecimento simbólico para tornar-se viável sem sanções; e (iii) a relativa escassez de estudos centrados em pais, sobretudo no setor de TI, é parte do problema e deve ser evidenciada empiricamente.

\section{Nota metodológica sobre o diagnóstico bibliográfico}
Para tornar essa lacuna mais visível, o Capítulo~\ref{chap:diagnostico} realiza um diagnóstico bibliográfico exploratório em bases de acesso direto. Comparam-se, de forma simétrica, as ocorrências, em português, das combinações ``maternidade e trabalho'' e ``paternidade e trabalho'' em três bases com perfis complementares: SciELO Brasil, Portal de Periódicos da CAPES e DOAJ (\emph{Directory of Open Access Journals}), no período de 2000 a 2025.

O procedimento resume-se a contar o número de registros para cada combinação de descritores em cada base e calcular a razão entre os conjuntos centrados em maternidade e em paternidade. A estratégia não pretende esgotar a produção existente, mas oferecer um indicador simples da sub-representação da paternidade na literatura sobre trabalho--família, articulando esses resultados exploratórios às discussões teóricas apresentadas neste capítulo.
