% ----------------------------------------------------------
\chapter{Introdução}
% ----------------------------------------------------------

\section{Contexto e justificativa}

O setor de Tecnologia da Informação (TI) opera sob a norma do ``profissional ideal'': disponibilidade contínua, atualização permanente e entregas constantes \cite{acker_1990}. Para homens que exercem a \emph{paternidade ativa}, esse padrão pressiona a conciliação entre o desempenho laboral e o cuidado cotidiano, produzindo tensões específicas entre os papéis de provedor, cuidador e trabalhador ``sem fricções'' \cite{greenhaus_beutell_1985,hochschild_machung_1989}. Tais tensões não são meramente individuais; são efeitos de arranjos de gênero que hierarquizam tarefas de cuidado e sustentam expectativas assimétricas sobre maternidade e paternidade \cite{connell_2005}.

No Brasil, a discussão pública e uma parcela relevante da literatura e das políticas institucionais tendem a associar a conciliação trabalho--família quase exclusivamente às mulheres, o que contribui para o \emph{apagamento} das dificuldades vividas por pais \cite{moser_pereira_2014}. Em um campo historicamente masculinizado como a TI, a combinação entre a figura do trabalhador sempre disponível e a naturalização do papel de provedor esvazia o reconhecimento da paternidade como responsabilidade de cuidado \cite{unesco_2017}. O \emph{gap} manifesta-se em dois níveis: (i) a relativa escassez de estudos centrados na experiência de pais trabalhadores --- especialmente em TI --- quando comparados ao vasto corpo sobre maternidade; e (ii) a persistência, em discursos cotidianos e organizacionais, de avaliações que deslegitimam ou minimizam a presença paterna no cuidado.

A motivação imediata deste trabalho surgiu de uma experiência pessoal, materializada em um desabafo publicado no LinkedIn sobre a dificuldade de sustentar o tripé família--estudos--trabalho na área de TI \cite{oliveira_linkedin_2025}. A repercussão da postagem, com múltiplos relatos de situações semelhantes, reforçou a percepção de que não se trata de um problema individual, mas de um fenômeno socialmente estruturado.

\section{Problema de pesquisa}

Quais efeitos do \emph{machismo estrutural} se manifestam nos conflitos de papéis vividos por pais que atuam no mercado de TI para conciliar responsabilidades profissionais e familiares, e de que modo essas dificuldades são invisibilizadas ou deslegitimadas no debate público e acadêmico --- particularmente nas interações em redes profissionais digitais?

\section{Objetivos}

\textbf{Objetivo geral}: investigar os conflitos de papéis entre provedor, cuidador e ``profissional ideal'' vividos por pais no mercado de TI, sob a lente do machismo estrutural.

\noindent\textbf{Objetivos específicos}:
\begin{itemize}
  \item revisar a literatura sobre conflito trabalho--família, masculinidades e paternidade ativa \cite{greenhaus_beutell_1985,connell_2005,lamb_2004,dermott_2008};
  \item mapear como políticas e normativas institucionais tratam (ou negligenciam) a paternidade no trabalho no contexto brasileiro \cite{ilo_2014,moser_pereira_2014};
  \item examinar discursos em rede profissional digital (LinkedIn) sobre paternidade e desempenho laboral em TI, com base em métricas e comentários de uma publicação autoral \cite{oliveira_linkedin_2025};
  \item propor agenda de práticas institucionais (licenças parentais, flexibilização temporal/espacial e cultura não punitiva à parentalidade masculina).
\end{itemize}

\section{Metodologia (resumo)}

Adota-se abordagem qualitativa triangulada: (i) revisão bibliográfica dirigida sobre conflito trabalho--família, masculinidades e paternidade; (ii) diagnóstico documental de políticas e normativas (legislação e diretrizes organizacionais) relativas à parentalidade no trabalho; e (iii) estudo de caso exploratório de interações em rede profissional digital, a partir de uma publicação autoral no LinkedIn, analisando conteúdo de comentários e métricas de engajamento (impressões, reações, compartilhamentos) \cite{oliveira_linkedin_2025}.

Para dimensionar a \emph{lacuna bibliográfica}, realizou-se um diagnóstico exploratório em cinco bases de acesso relativamente direto para estudantes de graduação: SciELO Brasil, Portal de Periódicos CAPES, DOAJ, Repositório Institucional da UFSC e ScienceDirect. Em todas elas aplicaram-se, de forma simétrica, combinações de descritores em português e em inglês (``maternidade e trabalho'' / ``motherhood and work'' e ``paternidade e trabalho'' / ``fatherhood and work''), no período de 2000 a 2025, contabilizando o número de registros e as proporções entre textos centrados na maternidade e na paternidade. Não se trata de uma revisão sistemática, mas de uma sondagem curta voltada a evidenciar a direção do desequilíbrio entre os dois conjuntos.

O recorte empírico foca o setor de TI. Os depoimentos oriundos do LinkedIn são apresentados de forma sintética, com preservação de anonimato quando necessário e atenção aos limites de representatividade de uma única rede social profissional.

\section{Estrutura do trabalho}

O Capítulo~2 apresenta o referencial teórico, discutindo gênero e divisão sexual do trabalho, paternidade ativa, machismo estrutural, conflito trabalho--família e características específicas do campo de TI. O Capítulo~3 realiza o diagnóstico do apagamento da paternidade ativa em políticas e literatura, articulando análise de marcos legais, panorama bibliométrico exploratório e o estudo das interações em rede profissional digital (LinkedIn). O Capítulo~4 traz as conclusões e considerações finais, retomando o problema de pesquisa, sintetizando os achados, discutindo implicações para o campo de TI e para Informática e Sociedade e indicando limitações e agenda futura.
