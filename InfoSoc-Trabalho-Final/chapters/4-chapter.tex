% ----------------------------------------------------------
\chapter{Conclusões e considerações finais}
\label{chap:conclusoes}
% ----------------------------------------------------------

\section{Retomada do problema e do percurso}

Este trabalho nasceu de um desabafo que publiquei no LinkedIn\cite{oliveira_linkedin_2025},
ao relatar a dificuldade concreta de sustentar o tripé família--estudos--trabalho
na área de TI. A rotina de cuidado com minha filha pequena passou a conviver com
a cobrança de disponibilidade quase ilimitada no emprego e nas entrevistas. A
resposta à postagem – com muitos comentários de pais e mães em situações
parecidas – mostrou que o problema ia além da minha experiência individual.

A partir daí, formulei a pergunta central: de que modo o machismo estrutural e o
ideal de trabalhador sempre disponível contribuem para apagar a paternidade
ativa, em especial na área de tecnologia da informação? Para discutir essa
questão, combinei análise de marcos legais e políticas de parentalidade, revisão
de literatura sobre normas organizacionais em TI, um diagnóstico bibliométrico
exploratório em bases de dados e o exame qualitativo das interações geradas por
essa postagem no LinkedIn\cite{oliveira_linkedin_2025}.

\section{Síntese dos achados e implicações}

Do ponto de vista jurídico, o ordenamento brasileiro protege de forma muito mais
robusta a maternidade do que a paternidade: licença-maternidade, estabilidade
gestante e políticas específicas estão consolidadas, enquanto a licença-paternidade
permanece curta e fragmentada \cite{cf_1988_art7,adct_art10_par1,
lei_11770_2008_empresa_cidada,decreto_8737_2016,lei_13257_2016_marco_primeira_infancia}.
No nível organizacional, a literatura sobre TI aponta a permanência do ideal de
\emph{trabalhador ideal} sempre disponível \cite{acker_1990}, e estudos recentes
indicam que homens que explicitam o cuidado ainda enfrentam desconfiança e
punições sutis \cite{mm360_paternidade_ativa,roberthalf_2023_paternidade_ativa}.

O diagnóstico bibliométrico em cinco bases (ScienceDirect, Portal de Periódicos
CAPES, DOAJ, SciELO Brasil e Repositório Institucional da UFSC) reforçou essa
assimetria: em todas elas há de três a oito textos sobre maternidade e trabalho
para cada texto sobre paternidade e trabalho. Em outras palavras, a experiência
de pais trabalhadores aparece como tema residual, mesmo em bases amplas e
internacionais. Os relatos da postagem no LinkedIn dão rosto a esses números:
pais que levam filhos ao médico, acompanham terapias ou reorganizam horários
relatam perda de oportunidades, cobranças de disponibilidade total e a sensação
de que precisam justificar o fato de cuidar.

Para o campo de TI e para Informática e Sociedade, esses achados indicam que
tecnologias, políticas de trabalho remoto e critérios de desempenho não são
neutros: articulam-se com normas de gênero que naturalizam o homem como provedor
sempre disponível e a mulher como cuidadora principal. Nesse contexto, exercer
paternidade ativa em TI ainda tende a ser visto como exceção, e não como forma
legítima de organizar a vida profissional.

\section{Limitações e agenda futura}

Este estudo é exploratório. As buscas bibliográficas concentraram-se em cinco
bases de acesso relativamente direto, com descritores simples em português e
inglês, o que pode ter deixado de fora produções que utilizam outras
terminologias ou recortes. Do ponto de vista qualitativo, a análise se baseia em
uma única postagem no LinkedIn, condicionada à minha rede de contatos e ao
algoritmo da plataforma, e não pretende representar a diversidade do setor de TI.

Ainda assim, os resultados sugerem caminhos para pesquisas futuras e para a
prática profissional: investigar políticas internas de empresas de tecnologia sob
a perspectiva dos pais cuidadores; comparar contextos nacionais com diferentes
regimes de licença parental; e discutir como arranjos de trabalho em TI podem
incorporar o cuidado – de mães e de pais – como uma dimensão legítima da
trajetória profissional, e não como exceção a ser tolerada.
